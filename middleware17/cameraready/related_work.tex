\section{Related work}
\label{sec:related}
\noindent
{\bf Hypervisor caching:} A hosted virtualization solution like 
KVM~\cite{kvm, kvmconfig} naturally offers \emph{inclusive}
hypervisor caching by virtue of being in the disk access path.
%
Singleton~\cite{singleton} performs cache translation of the
inclusive hypervisor cache to provide exclusive caching by deduplicating
KSM scanned pages and the inclusive cache.
%in an indirect manner.
%
Transcendent memory~\cite{memtrans} proposed a guest OS supported
exclusive caching framework which was
first implemented in Xen~\cite{oracletmem}.
%  
Exclusive hypervisor cache for KVM ~\cite{kvmzcache} based on the 
transcendent memory model enabled several caching combinations and 
%is comparatively
%analyzed to study the effectiveness of the hypervisor caching solutions.
analyzed effectiveness of the hypervisor caching solutions.
%  
All of the above solutions provide hypervisor cache management capability
at a VM granularity whereas \dd{} extends the framework to support application
level differentiated hypervisor cache partitioning in derivative
cloud setups.
%like for
%application containers hosted inside the VMs.
%
Mortar~\cite{mortar} exposed the hypervisor cache to distributed memory store 
applications like \textrm{memcached}~\cite{memcached} by modifying the application in an
explicit manner.
%
Our approach is more generic and does not require any modifications to 
applications.
%
Software-defined caching ~\cite{sdc} proposed application SLA guided 
dynamic second chance inclusive cache provisioning by hosting applications 
on separate virtual disks.
%
\dd{} provides additional flexibility in terms of hypervisor cache provisioning
with exclusive cache support and by providing different second chance storage options.
%
Moreover, our approach is better equipped to meet the decentralized memory 
management requirements of a derivative cloud setup as shown 
in \S\ref{subsec:sdc} and \S\ref{subsec:coop}.
%while the approach in ~\cite{sdc} is proposed for 
%multi-VM setups, we believe that the approach is in-principle 
%applicable to application containers as well;
%we provide a realization of the same. 

\noindent
{\bf Application containers and derivative clouds:} Linux \cgroups~\cite{cgroup},
BSD jails~\cite{jail} and Solaris Zones~\cite{zone} provide operating system 
level isolation support for different applications.
%  
LXC~\cite{lxc} and Docker~\cite{docker} are two popular application container
enablers built by exploiting the Linux \cgroup{} resource control framework.
%
Nested virtualization~\cite{turtle,blanket,recursv} enables hosting VMs inside a VM
by exposing hardware virtualization features to the first level VM.
%
The goal of nested virtualization has been to address security issues~\cite{vx32},
VM standardization related problems and hypervisor trouble-shooting. 
%
Resource management in nested virtualization setups has been proposed 
by ~\cite{intercloud} and ~\cite{supercloud}.
%
Intercloud~\cite{intercloud} and Supercloud~\cite{supercloud} propose 
interoperable 
cloud services 
decoupled from the cloud provider. 
%\puru{Add a line about intercloud and supercloud functionality}.
%by inter-clouds~\cite{intercloud} and super-cloud~\cite{supercloud}.
%
Derivative clouds in the current era~\cite{spotcheck, heroku, picloud} use 
containers inside the VMs as the IaaS primitive because virtualization 
overheads using containers is significantly lower than 
nested VM virtualization.
%
\dd{} enhances the provisioning flexibility through hypervisor cache management
at the nested container level and opens up resource-based SLA
business model enhancements for 
derivative clouds.


\noindent
{\bf Memory management in virtualized systems:} Memory 
ballooning~\cite{vmware,hotplug} facilitates dynamic adjustment of VM memory  
allocations.
%
Several dynamic ballooning based controllers to estimate and adjust the memory 
allocation to the VMs have been proposed~\cite{wss, membal, tws}.
%
Hypervisor caching and \dd{} complements the dynamic memory controllers
by providing symbiotic resource management capabilities.
%
\dd{} extends the flexibility of memory resource management in container 
within VM setups.
%
Memory deduplication~\cite{vmware,ksmpaper,satori} is another memory
efficiency enhancement technique shown to be useful in virtualized
systems~\cite{ksmpaper,utc}.   
%
Integration of \dd{} with deduplication to enhance the memory management
efficiency in a derivative cloud setup would be an interesting future
direction of our work.
%\puru{can we add another line about what this integration would mean,
%or what it implies for design? this is preceding line.}
%
%\revised{
While deduplication of hypervisor cache itself can be easily incorporated
into \dd{}, deduplication across hypervisor cache and memory used by VMs requires
additional techniques.
%While exclusive caching takes care of deduplication of objects in
%%of memory across 
%the hypervisor cache and page cache of individual virtual machines, there is 
%a chance of content duplication across virtual machines and the hypervisor
%cache.
%
Synergy~\cite{synergy}, a holistic deduplication technique in virtualized systems,
combines benefits of sharing and ballooning for resource provisioning
via the hypervisor cache. 
%
%\deba{cant comprehend the next sentence}
With the two-level storage option of \dd{}, 
shareable and non-shareable pages can use different storage options
to improve performance. We see this as yet another complimentary 
direction for future work.
%}
%to different treatment of shared and unshared pages, and is yet another
%interesting direction of future work.
%
% can be 
%incorporated into DoubleDecker to address the issue of duplication and left as
%a future direction. 
%}
%

